\documentclass{article}
\usepackage[utf8]{inputenc}
\usepackage{graphicx}
\usepackage{lipsum}

\begin{document}

\section{Strategic analysis}

\lipsum[1]

\section{Actors}

\lipsum[2]

\section{What is the project about?}
The project is about a self-service parcel pickup service.

\section{What is the goal of the project?}
The goal of the project is to provide a convenient and secure way for customers to pick up their packages at a designated location.

\section{Who is the target audience for the project?}
The target audience for the project is anyone who receives packages and wants a convenient and secure way to pick them up.

\section{What are the demographics of the target audience?}
The demographics of the target audience may vary, but they are likely to be urban or suburban residents who are busy and receive frequent packages. They may be of any age, gender, and socioeconomic status.

\section{What are the pain points and needs of the target audience?}
The pain points and needs of the target audience may include:
\begin{itemize}
    \item Difficulty tracking package deliveries and pickup times
    \item Lack of real-time updates on package status and locker availability
    \item Limited communication options with the pickup service
    \item Confusing or unreliable user interfaces for scheduling pickups and retrieving packages
    \item How can the project help address these pain points and needs?
    \item The project can help address these pain points and needs by providing a user-friendly web application that allows customers to track their package status, receive real-time updates on locker availability, communicate with customer support, and schedule pickups and deliveries. The application should also have a clear and intuitive user interface that simplifies the process of retrieving packages from the locker.
\end{itemize}

\section{What are some common characteristics, behaviors, and preferences of the target audience?}
The target audience may be tech-savvy and prefer digital communication and services. They may value convenience, security, and flexibility in their package delivery and pickup options. Additionally, they may prefer mobile and web applications that are easy to use and provide a seamless user experience.

\section{What are some key factors that may influence the target audience's decision-making process?}
Key factors that may influence the target audience's decision-making process may include the reliability and responsiveness of the application, the security and convenience of the service, and the cost of the service.

\section{How can the project leverage these factors to achieve its goal?}
The project can leverage these factors by offering a reliable, secure, and affordable web application that provides real-time updates, clear communication channels, and a seamless user experience. The application should also be optimized for mobile devices, as many customers may prefer to use their smartphones to track their packages and schedule pickups.

\section{What are some possible scenarios in which the target audience may interact with the project, and how can the project provide a positive user experience in these scenarios?}
Some possible scenarios in which the target audience may interact with the project through the web application include:

\begin{itemize}
    \item Tracking the status of a package delivery
    \item Scheduling a pickup time
    \item Receiving real-time updates on locker availability
    \item Retrieving a package from the locker
\end{itemize}

To provide a positive user experience, the web application should be designed with a clear and intuitive user interface that simplifies the process of scheduling pickups and retrieving packages from the locker. Additionally, the application should provide real-time updates on package status and locker availability, and offer clear communication channels for customer support. The application should also be optimized for mobile devices to provide a seamless user experience for customers who prefer to use their smartphones.

\section{Personas}

Name: Sarah

Age: 32

Occupation: Marketing Manager

Location: Urban area

Education: Bachelor's degree in marketing

Goals: Sarah wants a hassle-free package pickup experience that is both secure and convenient. She wants to be able to pick up her packages at a time that suits her schedule without having to worry about package theft or loss. She values a service that is easy to use and provides clear communication channels for customer support.

Challenges: Sarah's work schedule is often unpredictable, which makes it difficult for her to be available during delivery hours. She may also have limited access to a car, which can make it challenging to pick up packages from a centralized location. Additionally, she is concerned about package theft, as she lives in an urban area where package theft is common.

Technology use: Sarah is tech-savvy and frequently uses her smartphone and computer for work and personal tasks. She prefers digital communication and services that are user-friendly and easy to use. She is comfortable using online payment systems and has experience using self-service applications.

Based on Sarah's persona, we can design the self-service parcel pickup service and web application to meet her needs by providing a secure, flexible, and user-friendly experience. This may include features such as:
\begin{itemize}
    \item Real-time updates on package status and locker availability
    \item Clear communication channels for customer support, including online chat or email support
    \item An intuitive user interface for scheduling pickups and retrieving packages, with the ability to select a pickup time that suits Sarah's schedule
    \item Mobile optimization for easy access and a seamless user experience on smartphones
    \item Secure and easy-to-use online payment system for any additional fees or charges
    \item A convenient and accessible pickup location, such as a locker located near Sarah's home or workplace
    \item Measures to ensure package security, such as video surveillance and secure access codes for locker retrieval
\end{itemize}


\newpage

\documentclass{article}
\usepackage[utf8]{inputenc}
\usepackage{graphicx}
\usepackage{lipsum}

\begin{document}
Name: John

Age: 68

Occupation: Retired

Location: Suburban

Education: Bachelor's degree in business

Goals: John wants a reliable and easy-to-use package pickup service that allows him to receive packages from his adult children living abroad. He values flexibility and convenience, as he may have limited mobility due to age-related health issues. Additionally, he wants a service that is easy to use and provides clear communication channels for customer support.

Challenges: John may have limited mobility due to age-related health issues, which can make it difficult for him to travel to centralized package pickup locations. Additionally, he may have limited experience using digital technology, which can make it challenging to navigate the self-service pickup process. Finally, he may also be concerned about package theft, as he lives in an area where package theft is common.

Technology use: John has some experience using digital technology, including email and basic internet browsing. However, he may not be comfortable using more advanced features or self-service applications. He prefers services that are user-friendly and straightforward.

Based on John's persona, we can design the self-service parcel pickup service and web application to meet his needs by providing a reliable, convenient, and user-friendly experience. This may include features such as:
\begin{itemize}
    \item A dedicated customer support team to assist John with any questions or issues he may have
    \item An intuitive user interface for scheduling pickups and retrieving packages, with clear and simple instructions
    \item A convenient and accessible pickup location, such as a locker located near John's home or workplace, with the ability to request home delivery if needed
    \item Mobile optimization for easy access and a seamless user experience on smartphones or tablets, with clear and simple instructions for use
    \item Measures to ensure package security, such as video surveillance and secure access codes for locker retrieval, with clear and simple instructions for use
\end{itemize}

\end{document}

\newpage 

\section{User Story Mapping}

\input{user-stories.tex}

\end{document}
